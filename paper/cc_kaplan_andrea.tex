\documentclass{article}\usepackage{graphicx, color}
%% maxwidth is the original width if it is less than linewidth
%% otherwise use linewidth (to make sure the graphics do not exceed the margin)
\makeatletter
\def\maxwidth{ %
  \ifdim\Gin@nat@width>\linewidth
    \linewidth
  \else
    \Gin@nat@width
  \fi
}
\makeatother

\IfFileExists{upquote.sty}{\usepackage{upquote}}{}
\definecolor{fgcolor}{rgb}{0.2, 0.2, 0.2}
\newcommand{\hlnumber}[1]{\textcolor[rgb]{0,0,0}{#1}}%
\newcommand{\hlfunctioncall}[1]{\textcolor[rgb]{0.501960784313725,0,0.329411764705882}{\textbf{#1}}}%
\newcommand{\hlstring}[1]{\textcolor[rgb]{0.6,0.6,1}{#1}}%
\newcommand{\hlkeyword}[1]{\textcolor[rgb]{0,0,0}{\textbf{#1}}}%
\newcommand{\hlargument}[1]{\textcolor[rgb]{0.690196078431373,0.250980392156863,0.0196078431372549}{#1}}%
\newcommand{\hlcomment}[1]{\textcolor[rgb]{0.180392156862745,0.6,0.341176470588235}{#1}}%
\newcommand{\hlroxygencomment}[1]{\textcolor[rgb]{0.43921568627451,0.47843137254902,0.701960784313725}{#1}}%
\newcommand{\hlformalargs}[1]{\textcolor[rgb]{0.690196078431373,0.250980392156863,0.0196078431372549}{#1}}%
\newcommand{\hleqformalargs}[1]{\textcolor[rgb]{0.690196078431373,0.250980392156863,0.0196078431372549}{#1}}%
\newcommand{\hlassignement}[1]{\textcolor[rgb]{0,0,0}{\textbf{#1}}}%
\newcommand{\hlpackage}[1]{\textcolor[rgb]{0.588235294117647,0.709803921568627,0.145098039215686}{#1}}%
\newcommand{\hlslot}[1]{\textit{#1}}%
\newcommand{\hlsymbol}[1]{\textcolor[rgb]{0,0,0}{#1}}%
\newcommand{\hlprompt}[1]{\textcolor[rgb]{0.2,0.2,0.2}{#1}}%

\usepackage{framed}
\makeatletter
\newenvironment{kframe}{%
 \def\at@end@of@kframe{}%
 \ifinner\ifhmode%
  \def\at@end@of@kframe{\end{minipage}}%
  \begin{minipage}{\columnwidth}%
 \fi\fi%
 \def\FrameCommand##1{\hskip\@totalleftmargin \hskip-\fboxsep
 \colorbox{shadecolor}{##1}\hskip-\fboxsep
     % There is no \\@totalrightmargin, so:
     \hskip-\linewidth \hskip-\@totalleftmargin \hskip\columnwidth}%
 \MakeFramed {\advance\hsize-\width
   \@totalleftmargin\z@ \linewidth\hsize
   \@setminipage}}%
 {\par\unskip\endMakeFramed%
 \at@end@of@kframe}
\makeatother

\definecolor{shadecolor}{rgb}{.97, .97, .97}
\definecolor{messagecolor}{rgb}{0, 0, 0}
\definecolor{warningcolor}{rgb}{1, 0, 1}
\definecolor{errorcolor}{rgb}{1, 0, 0}
\newenvironment{knitrout}{}{} % an empty environment to be redefined in TeX

\usepackage{alltt}

\begin{document}

\part{Background}
\section{Networks}
\section{Visualization}
\section{Layout Algorithms}

\part{User Interface}
\section{Technical Aspects}
\subsection{Software}
This tool utilizes three main pieces of software to establish interactive user control of a random graph. The three pieces used are Shiny, D3, and igraph. They are used to manage server/client interaction, user interface, and data formatting, respectively.


\subsubsection{Shiny}
Shiny is an R package created by RStudio that enabled R users to create an interactive web application that utilizes R as the background engine. Through default methods to build user interface elements in HTML and a handle to the server side code, Shiny is a very simple way to turn R code into a website. 

This tool uses the Shiny functionality to create user controls, pass correctly formatted data to the client (and thus usable by javascript), and as a means to display summary information regarding the user's interactions with a graph at any point in time. In this context, Shiny serves as the translator between the formatted data and what the user sees and interacts with on their screen.


\subsubsection{D3}
D3 is a javascript library written by Mike Bostock with the main purpose of visualizing and interacting with data. The title D3 stands for ``Data Driven Documents”. This library facilitates manipulation of HTML elements, SVG (scalable vector graphics), and CSS (cascading style sheets) with the end goal of animations and interactions that are tied to underlying data. The key idea behind the library is that DOM elements are completely determined by the data, so rather than adding elements to a web page to be viewed by users, D3 allows users to see and interact with graphical representaions of their data in a web framework. 

This tool uses D3 to handle all graphical displays and user interactions with the graph. The data is passed to the client and able to be used through Shiny's input bindings. At this point, the graph’s nodes are tied to circles and the edges are tied to paths on the page. User manipulations such as selecting, dragging, and grouping are handled by D3 and then data is passed back to the server via Shiny's output bindings to allow for communication between user and the R engine underneath.


\subsubsection{igraph}
igraph is a software package used for creating and manipulating undirected and directed graphs. It is a cross-language packagae available for C, R, python, and Ruby. igraph also supports multiple graph file formats and visualization of graph structures.

This tool utilizes two parts of igraph, first is the conversion from a gml file to an xml file. The gml file format, short for Graph Modelling Language,is a hierarchical ASCII-based file format for describing graphs. Then, once we have an xml file we can convert to a JSON file using the R package rjson. Getting the graphs in a JSON file formats makes working with them in the D3 library incredibly straightforward. 

igraph is also used to compute initial x and y coordinates for the graph using a force-driven layout prior to being passed to the force layout in D3. This reduces the initial work that must be done by the javascript library and helps minimize unnecessary movement by the nodes. This is critical as the extra movement at the loading of the pages creates an unnecessarily chaotic start to the user's experience.



\subsection{Data Management}
\section{Design and Functionality}

\part{Further Work}


\end{document}
